%////////////////////////////////////////////////////////////////////////////////////////////////////////////
% COSTA RICA INSTITUTE OF TECHNOLOGY
% Material Science and Engineering School
% https://www.tec.ac.cr
% Link on Github: https://github.com/Sentropic/Plantilla-Articulo-ME
% Sebas Solís Vargas (they/them)
% Email: sentropic.sv@gmail.com
% https://twitter.com/Sentropic_
% https://github.com/Sentropic

%/////////////////////////////////////////////////////////////////////////////////////////////////////////////
% BLOCK: LATEX document configuration and packages loading
%/////////////////////////////////////////////////////////////////////////////////////////////////////////////

% Sheet format and letter size is defined
\documentclass[10pt, a4paper, twocolumn]{article} 
\usepackage[fontsize=9pt]{fontsize}
\setlength{\parindent}{0pt}
\setlength{\columnsep}{4em}

% Date commands are defined
\def\mydate{\leavevmode\hbox{\the\year/\twodigits\month/\twodigits\day}}
\def\twodigits#1{\ifnum#1<10 0\fi\the#1}

% Margens are defined
\usepackage[left=2.26cm,right=2.08cm,top=2.01cm,bottom=2.01cm]{geometry} 
\setlength{\marginparwidth}{1.8cm}

%Translates various standard and input encodings into LaTeX internal language
\usepackage[utf8]{inputenc}
\usepackage[T1]{fontenc}
\usepackage[lining]{ebgaramond}

% Several mathematical symbolologies are defined
\usepackage{amsmath} 
\usepackage{amsfonts} 
\usepackage{amssymb}

% Mathematical equations are enumerated
\usepackage{makeidx}

% Packages for Excel2Latex are loaded
\usepackage{bigstrut}

% Document language is selected
\usepackage[english]{babel}
\usepackage{csquotes}

% Document spacing is defined
\usepackage{setspace}

% Document spacing is set
\onehalfspacing %Spacing

% Improves interface for floating objects (figures and tables)
\usepackage{float} 

% Allows manipulation of images, headers, footers, etc.
\usepackage{graphicx}
\usepackage{fancyhdr}
\usepackage{titling}
\usepackage{titlesec}
\titleformat{\section}[hang]{\fontsize{10}{12} \bfseries}{\thesection}{2em}{}
\titlespacing{\section}{0pt}{0.5em}{0.5em}

% Bibliography packages are loaded
\usepackage[backend=biber,sorting=none]{biblatex}
\usepackage[hidelinks]{hyperref}

% References (*.BIB) file is loaded !!!
\addbibresource{reference.bib}

% Allows adding references within the index
\usepackage[nottoc]{tocbibind}

% Allows temporary annotations during writing
% Can be disabled adding [disable]
\usepackage[textsize=tiny]{todonotes}

%/////////////////////////////////////////////////////////////////////////////////////////////////////////////
% BLOCK: LATEX document writing
%/////////////////////////////////////////////////////////////////////////////////////////////////////////////

% Document is started
\begin{document}

\twocolumn[
\begin{minipage}{0.595\textwidth} % Left side of header
\raggedright
\raisebox{-0.38\height}{\includegraphics[scale=1]{images/template/ME logo.png}} {\fontsize{9}{10} \textbf{Material Science and Engineering School}}
\end{minipage}
\begin{minipage}{0.395\textwidth} % Right side of header
\raggedleft
\raisebox{-0.3\height}{\includegraphics[scale=1]{images/template/TEC logo.png}}
\end{minipage}

\begin{center}
	\fontsize{14}{18} \textbf{<TITLE>}
\end{center}

\fontsize{11}{14} \textbf{<Autor>$^1$, <Autor>$^2$}\medskip\\
<Course ID> <Course name>\\
<Semester> Semester \the\year\\
Date: \mydate\\
<Prof.|Dr.|Eng.|etc> <Professor>\medskip\\
$^{1,2}$ Material Science and Engineering School, Costa Rica Institute of Technology (ITCR), Cartago 159-7050, Costa Rica\\
$^1$ \href{mailto:email@uni.edu}{email@uni.edu}, $^2$ \href{email@uni.edu}{email@uni.edu}\bigskip\\
\fontsize{9}{12} \textbf{Keywords: <Keyword>, <Keyword>, <Keyword>, <Keyword>, <Keyword>}

\begin{center}
\fontsize{10}{12} \textbf{Abstract}
\end{center}
\renewcommand{\abstractname}{\vspace{-2.5em}}
\begin{abstract}
As a primary goal, the abstract should render the general significance and conceptual advance of the work clearly accessible to a broad readership. References should not be cited in the abstract. 
\end{abstract}\bigskip]

\section{Introduction}
For Original Research Articles, Clinical Trial Articles, and Technology Reports the introduction should be succinct, with no subheadings.

\section{Materials and Methods}
This section may be divided by subheadings. This section should contain sufficient detail so that when read in conjunction with cited references, all procedures can be repeated.

\section{Results and Discussion}
This section may be divided by subheadings. Footnotes should not be used and have to be transferred into the main text.
Discussions should cover the key findings of the study: discuss any prior art related to the subject so to place the novelty of the discovery in the appropriate context; discuss the potential short-comings and limitations on their interpretations; discuss their integration into the current understanding of the problem and how this advances the current views; speculate on the future direction of the research and freely postulate theories that could be tested in the future \cite{cengel, coleparmer, goulds}.
Please note that the Material and Methods section can be placed in any of the following ways: before Results, before Discussion or after Discussion.

\begin{figure}[htbp!]
	\centering
	\missingfigure{Replace placeholder figure}
	\caption{You can add temporary figures}
	\label{fig:placeholder}
\end{figure}

\section{Conclusions}
Note here the remarks of your work. By reading the abstract and conclusion any person on the field must be able to understand your work.

\section*{Acknowledgements}
This is a short text to acknowledge the contributions of specific colleagues, institutions, or agencies that aided the efforts of the authors. 
Funding: Details of all funding sources should be provided, including grant numbers if applicable. Please ensure to add all necessary funding information, as after publication this is no longer possible.

\section*{References}
\printbibliography[heading=none]

%/////////////////////////////////////////////////////////////////////////////////////////////////////////////
% BLOCK: To-do list (REMOVE BEFORE FINAL COMPILING)
%/////////////////////////////////////////////////////////////////////////////////////////////////////////////

% Displays to-do list
\listoftodos
\href{http://tug.ctan.org/macros/latex/contrib/todonotes/todonotes.pdf}
{\textbf{Click to go to the todonotes manual}}
\todo{Remove todo list}

\end{document}
